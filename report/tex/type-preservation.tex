\begin{theorem}[Type preservation for expressions]\label{pres-expPreservation}
  If $\Gamma\vdash \mathtt{e}:T\ |_C\ \Gamma'$ and $\Gamma\vdash(s, \theta)$ and we have some
  transition $\transition{e}{s}{\theta}{e'}{s'}{\theta'}$, then some $\gamma$ can augment $\Gamma$ such that
  $\Gamma\!\cup\!\gamma\vdash\mathtt{e'}:T\ |_C'\ \Gamma''$ and $\Gamma\!\cup\!\gamma\vdash(s,\theta)$ and
  $\Gamma'\sqsubseteq\Gamma''$.
\end{theorem}

\begin{proof}
  The proof is by rule induction over over all transition rules for
  expressions. Take 

  \begin{multline*}
 	\Phi(e, s, \theta, e', s', \theta') \eqdef 
	\	\forall\Gamma,C,\Gamma' \qdot
	\Gamma\vdash\mathtt{e}:T\ |_C\ \Gamma' 
	\enspace\land\enspace\Gamma\vdash (s,\theta)
	\implies \\
   	(\exists\gamma,\Gamma'',C'\qdot
    \Gamma\!\cup\!\gamma\vdash \mathtt{e'}:T\ |_{C'}\ \Gamma'' 
	\enspace\land\enspace\Gamma\!\cup\!\gamma\vdash (s',\theta')
	\enspace\land\enspace \Gamma'\sqsubseteq\Gamma''),
  \end{multline*} 
  where $C$ and $C'$ are satisfiable sets of constraints. We show that for all
  $e, s, \theta, e', s', \theta'$, if there exists some transition
  $\transition{e}{s}{\theta}{e'}{s'}{\theta'}$, then
  $\Phi(m,s,\theta,m',s',\theta')$ holds. We show this by considering all
  possible last steps in the derivation of the transition judgement. In all
  cases, we begin by considering arbitrary $\Gamma, C$ and $\Gamma'$, and
  assuming that $\mathtt{e}$ is typable, has type $T$, and that the store is well-typed. This leaves us to find
  a suitable $\gamma$ and to show the following three things to prove $\Phi(e,s,\theta,e',s',\theta')$:

  \begin{gather}
	\Gamma\!\cup\!\gamma\vdash\mathtt{e'}:T\ |_{C'}\ \Gamma'' \label{pres-type1}\\
	\Gamma\!\cup\!\gamma\vdash(s',\theta') \label{pres-store1} \\
	\Gamma' \sqsubseteq \Gamma'' \label{pres-env1}.
  \end{gather}

  For expressions, we require a slightly stronger statement for~\eqref{pres-type1} -- that
  the judged type of $\mathtt{m'}$ be the same as the judged type for $\mathtt{m}$.
  I only include below a sketch of the full proof, though the results should for the
  most part be clear.

  \begin{case}[Func]

	The conclusion is immediately typable by \textsc{V\_Closure} using $\gamma=\Gamma$. The domain
	precondition is satisfied since $\Gamma\vdash(s,\theta)$, and the second by assumption.
  \end{case}
  \begin{case}[Obj, Arr]
	Using the induction hypothesis, we know that $e'$ is well-typed. As a result,
	the preconditions are satisfied for \textsc{V\_Obj} or \textsc{V\_Arr}.
  \end{case}
  \begin{case}[Deref, Prop2, ArrGet3, VRef]
	$e'$ takes the form of a single value after this transition, and the store
	is not modified. All values are typable under any type environment, and
	so $\Phi$ holds.
  \end{case}
  \begin{case}[Prop1, ArrGet1, ArrGet2, Call1, Call2, CallBody1, Assign1, Assign2]
	After the transition, $e'$ has the same form as $e$, and so 
	is typable using the same rule and type environment after using the induction
	hypothesis to ensure that all subexpressions are typable. We can also use 
	the induction hypothesis on the premise of the transition rule to
	establish that $\Gamma\vdash(s',\theta')$.
  \end{case}
  \begin{case}[CallNamed, CallAnon, PropCallNamed, PropCallAnon]
	In order for the closure to be deemed typable, we must have that, for some $\gamma$
	$$\gamma\vdash func:[\![T,\gamma]\!]\ |_C\ \gamma'$$
	from rule \textsc{V\_Closure}. This in turn can only have been derived through 
	one of \textsc{AnonVoid}, \textsc{AnonFun}, \textsc{NamedFun} or \textsc{NamedVoid},
	which all require that some extension of $\gamma$ be able to type the function body.
	This condition is all that is required for rule \textsc{FunBodyType} or \textsc{VoidBodyType},
	which can be used to derive a type judgement for $e'$.

	The type environment $\Gamma$ is not changed, and each of the additions to the store
	involve fresh types which will not occur in $\Gamma$, and so we can still say 
	that $\Gamma\vdash(s',\theta')$.
  \end{case}
  \begin{case}[CallBody2, CallBody3, CallBody4]
	Similar to case \textsc{Deref}.
  \end{case}
  \begin{case}[Assign3]

	Syntactically, there are three rules which could have determined that
	$\mathtt{e}$ is typable -- \textsc{AssignType}, \textsc{AssignTypeUndef} or
	\textsc{PropAssignType}. 

	If the rule used was \textsc{AssignTypeUndef}, then $\mathtt{vRef}$ would
	have to be of the form `$\mathtt{id}$'. The precondition, that $(vRef,s,
	\theta)\in dom(addr)$, can only hold if $id\in dom(s)$. Since
	$\Gamma\vdash(s,\theta)$ this would mean that $id\in\Gamma$. This would
	violate a precondition of \textsc{AssignTypeUndef}, so we can eliminate
	this rule as a possibility.

	Let $\theta'=\theta\oplus\{addr(vRef,s,\theta):v\}$ in the following discussion.

	\begin{subcase}[AssignType]
	  We will again use an empty $\gamma$ for this case.
	  An inspection of the possible typing rules for the expression `$\mathtt{vRef}$'
	  (\textsc{IdType, PropType} or \textsc{ArrayType}), shows that there can
	  $\Gamma_1$ and $\Gamma$ must be equal.
	  We can then deduce~\eqref{pres-type1} from the second precondition of \textsc{AssignType}:
	  $$\Gamma\vdash\mathtt{e'}:T_2\ |_{C_2}\ \Gamma_2.$$ 
	  We can also determine that $\Gamma=\Gamma_2$, since $\mathtt{e'}$ is a value, and
	  so~\eqref{pres-env1} is satisfied. By assumption, $\Gamma\vdash(s,\theta)$, so
	  we just need to show that
	  $\Gamma$ can handle the addition of the new heap address for~\eqref{pres-store1}.

 	  The scope is unchanged, and so $dom(\Gamma)\subseteq dom(s)$ remains true.
	  We know that $\Gamma\vdash\mathtt{vRef}:T_1\ |_{C_1}\ \Gamma$ from the premise of 
	  \textsc{AssignType}. We also know that $addr(vRef, s, \theta')$ is 
	  well defined, and that $\theta'(addr(vRef, s, \theta')) = v$ 
	  from the transition rule \textsc{Assign3}. We have already determined
	  that $\mathtt{v}$ has type $T_2$, and the constraints of \textsc{AssignType}
	  give us that $T_1\succeq T_2$. Hence $\Gamma\vdash(s,\theta')$ is satisfied,
	  and $\Phi(\mathtt{e},s,\theta,\mathtt{e'},s',\theta')$ holds.
  	\end{subcase}

	\begin{subcase}[PropAssignType]
	  The complication of this case comes from
	  the fact that the output type environment, $\Gamma_1$, is larger than the input, $\Gamma$.
	  To handle this, then, we do not use an empty $\gamma$, but instead use
	  $$\gamma=\{\textbf{id}: \{ \mathbf{l_1}: \{\cdots \{\mathbf{l_k}:\{\mathbf{l}: T'\}\orig T_k \} \cdots\} \orig T_1\}\orig T_0\}.$$
	  $\Gamma\!\cup\!\gamma=\Gamma_1$, and so~\eqref{pres-env1} is trivially
	  satisfied. We can also show that~\eqref{pres-type1} holds because $\mathtt{m'}$ is a value.
	  $$\Gamma\!\cup\!\gamma\vdash\mathtt{v}:T'\ |_{C'}\ \Gamma\!\cup\!\gamma$$
	  We now must show that $\Gamma\!\cup\!\gamma\vdash(s,
	  \theta')$, for which we use the assumption that
	  $\Gamma\vdash(s,\theta)$. Note that
	  $dom(\Gamma)=dom(\Gamma\cup\gamma)$, since the first precondition
	  of \textsc{PropAssignType} requires that $id\in dom(\Gamma)$.

	  For the value reference $\mathtt{id.l_1.\cdots.l_k.l}$, clearly we have that
	  $$\Gamma\!\cup\!\gamma\vdash\mathtt{id.l_1.\cdots.l_k.l}:T'\ |_\varnothing\ \Gamma\!\cup\!\gamma,$$
	  which is of the right type to satisfy the conclusion
	  of~\eqref{eqn:typed-store1}, since
	  $$\theta'(addr(\mathtt{id.l_1.\cdots.l_k.l},s, \theta')) = \mathtt{v},$$
	  and $\mathtt{v}$ also has type $T'$.
	  For sub-references of $\mathtt{id.l_1.\cdots.l_k.l}$ (such as $\mathtt{id.\cdots.l_i}$), we will have
	  \begin{gather*}
		\Gamma\!\cup\!\gamma\vdash\mathtt{id.\cdots.l_i}:\{l_{i+1}:T_{i+1}\}\orig T_i\ |_\varnothing\ \Gamma\!\cup\!\gamma \\
		\theta'(addr(\mathtt{id.\cdots.l_i},s,\theta'))	= \{l_{i+1}:a_{i+1}, p_0:a_0,\cdots,p_k:a_k\} \\
		\Gamma\!\cup\!\gamma\vdash\mathtt{\{l_{i+1}:a_{i+1}, p_0:a_0,\cdots,p_k:a_k\}} 
		:\{l_{i+1}:T_{i+1}', p_0:T_{p_0},\cdots p_k:T_{p_k}\}\ |_\varnothing\ \Gamma\!\cup\!\gamma
	  \end{gather*}
	  We want to show that $\{l_{i+1}:T_{i+1}\}\orig T_i\succeq \{l_{i+1}:T_{i+1}', p_0:T_{p_0},\cdots p_k:T_{p_k}\}$, which we can 
	  do by splitting the two parts of the origin chain on the left, and demonstrating that both
	  are supertypes.

	  Because $\Gamma\vdash(s,\theta)$, we must have that $T_i\succeq\{p_0:T_{p_0},\cdots p_k:T_{p_k}\}$, and so 
	  $$T_i\succeq\{l_{i+1}:T_{i+1}', p_0:T_{p_0},\cdots p_k:T_{p_k}\}.$$
	  $\{l_{i+1}:T_{i+1}\}\succeq \{l_{i+1}:T_{i+1}', p_0:T_{p_0},\cdots p_k:T_{p_k}\}$ clearly
	  holds as long as $T_{i+1}\succeq T_{i+1}'$ does. In fact, we can show that this does hold
	  by considering the sub-reference $\mathtt{id.\cdots.l_i.l_{i+1}}$ as an induction
	  step, using $\mathtt{vRef}$ as the base case.

	  Finally, for all other value references, $\gamma$ is not involved, and
	  nor is the extension to $\theta$, and so
	  $$\Gamma\vdash(s,\theta)\implies\Gamma\!\cup\!\gamma\vdash(s,\theta')$$ 
	  \eqref{pres-store1} is satisfied and $\Phi(\mathtt{e},s,\theta,e',s',\theta')$ holds.

	\end{subcase}
  \end{case}

  \begin{case}[AssignUndef]
	Similar to the first subcase of \textsc{Assign}, except that $\gamma$ is
	not empty - it contains the entry $\{id: T_v\}$.
  \end{case}
  \begin{case}[PropAdd]
	See the second subcase of \textsc{Assign}.
  \end{case}
  \begin{case}[AssignTarget1, AssignTarget2, AssignTarget3]
	In each of these cases, $e'$ is of the same form as $e$ -- both are an
	assignment target. The same rule is able to type $e'$ as is $e$, using the
	induction hypothesis conclusion as premise. The induction hypothesis also
	gives us that $\Gamma\vdash(s',\theta')$.
  \end{case}
  \begin{case}[BinOp1, BinOp2, PreAssign, PostAssign]
	In both cases, $e'$ is of the same form as $e$, and the type judgement for $e'$ 
	can be derived using the same rules as $e$. The precondition for the last 
	rule in the derivation tree is satisfied using the induction hypothesis.
	The induction hypothesis also gives us that the new store is well-typed.
  \end{case}
  \begin{case}[NumOp, BoolOp, CmpOp, NumCmpOp, PreOp, BoolNeg, NumNeg, PostInc, PreInc, PostDec, PreDec]
	In all cases $e'$ is a value, which is trivially typed by any $\Gamma$.
	The store is unchanged, and so we still have that $\Gamma\vdash(s,\theta)$
	by assumption.
  \end{case}
\end{proof}





\begin{theorem}[Type preservation for statements]\label{pres-mPreservation}
  If $\Gamma\vdash \mathtt{m}\ |_C\ \Gamma'$ and $\Gamma\vdash(s, \theta)$ and we have some
  transition $\transition{m}{s}{\theta}{m'}{s'}{\theta'}$, then some $\gamma$ can augment $\Gamma$ such that
  $\Gamma\!\cup\!\gamma\vdash\mathtt{m'}\ |_C'\ \Gamma''$ and $\Gamma\!\cup\!\gamma\vdash(s,\theta)$ and 
  $\Gamma'\sqsubseteq\Gamma''$.
\end{theorem}
\begin{proof}
  \begin{multline*}
 	\Phi(m, s, \theta, m', s', \theta') \eqdef 
	\	\forall\Gamma,C,\Gamma' \qdot
	\Gamma\vdash\mathtt{m}\ |_C\ \Gamma' 
	\enspace\land\enspace\Gamma\vdash (s,\theta)
	\implies \\
   	(\exists\gamma,\Gamma'',C'\qdot
    \Gamma\!\cup\!\gamma\vdash \mathtt{m'}\ |_{C'}\ \Gamma'' 
	\enspace\land\enspace\Gamma\!\cup\!\gamma\vdash (s',\theta')
	\enspace\land\enspace \Gamma'\sqsubseteq\Gamma''),
  \end{multline*} 


  \begin{case}[Seq1]
	In order to type $e$, we must have that each of the statements $m_1,\dots,m_i$ must
	be individually typable, which is all that is required to show that $e'$ 
	is typable. The store is unchanged, and hence $\Gamma\vdash(s',\theta')$.
  \end{case}
  \begin{case}[Seq2]
	In this case, the suitable $\gamma$ is the empty type environment, so 
	$\Gamma\!\cup\!\gamma=\Gamma$. $\mathtt{m}$ must
	be typable under rule \textsc{SeqTypable}, which requires as premise that
	each $\mathtt{m_1}, \dots, \mathtt{m_i}$ themselves be typable. In particular, we
	require that 
	$$\Gamma\vdash\mathtt{m_1}\ |_{C_1}\ \Gamma_1,$$ 
	and using the assumption that $\Gamma\vdash(s,\theta)$ with the induction hypothesis gives us that
	\begin{gather*}
	  \Gamma\vdash\mathtt{m_1'}\ |_{C}\ \Gamma_1' \\
	  \Gamma\vdash(s',\theta') \\
	  \Gamma_1\sqsubseteq\Gamma_1'.
  	\end{gather*}
  	This gives us~\eqref{pres-store1} directly. Using
	Lemma~\ref{lm:strChain}, and knowing that $\Gamma_1\vdash\mathtt{m_2}\ |_{C_2}\ \Gamma_2$
   	we can determine that 
	\begin{gather*}
	  \Gamma_1'\vdash\mathtt{m_2}\ |_{C_2}\ \Gamma_2' \\
	  \Gamma_2\sqsubseteq\Gamma_2'.
  	\end{gather*}
	We can repeat this for each statement $\mathtt{m_3\dots m_i}$, eventually ending up with
	\begin{gather*}
	  \Gamma_{i-1}'\vdash\mathtt{m_i}\ |_{C_i}\ \Gamma_i' \\
	  \Gamma_i\sqsubseteq\Gamma_i'
	\end{gather*}
	Which gives us~\eqref{pres-env1}, and satisfies all the premises for rule
	\textsc{SeqTypable}. We can then conclude that 
	$$\Gamma\vdash\mathtt{m_1'; \dots; m_i}\ |_{\bigcup_{k=0}^i{C_k}}\ \Gamma_i',$$
	giving us~\eqref{pres-type1}, and $\Phi(m,s,\theta,m',s',\theta')$ is satisfied.

  \end{case}
  
  \begin{case}[Var1, Var3]
	$m'$ is of the form $\epsilon$, which is trivially typable by any type environment.
	The proof that $\Gamma\vdash(s',\theta')$ is very similar to the proof for
	the case \textsc{AssignUndef}.
  \end{case}
  \begin{case}[Var2, Return1, If1]
	$m'$ has the same form as $m$, and so we can use the same rule to derive its
	typability, using the induction hypothesis to show that $e'$ is still typable.
	The induction hypothesis also gives us that $\Gamma\vdash(s',\theta')$.
  \end{case}
  \begin{case}[Var3]
	From rule \textsc{MultiDecTypable}, we immediately have that $m'$ is still typable after
	a transition. The store is unchanged, and so we still have that $\Gamma\vdash(s,\theta)$ 
	from our initial assumption.
  \end{case}
  \begin{case}[FuncDef]
	Similar to \textsc{Var3}.
  \end{case}
  \begin{case}[Return2]
	\js{return undefined;} is immediately typable using rule \textsc{RetTypable4}. 
	The store is unchanged, so we still have $\Gamma\vdash(s,\theta)$ from our initial
	assumption.
  \end{case}

  \begin{case}[If2]

	We again use $\gamma=\{\}$ here. The only type judgement which could result in $\typable{m}$ uses rule
	\textsc{IfTypable1}. One precondition for \textsc{IfTypable1} is that
	$$\Gamma\vdash\mathtt{e}:T_0\ |_{C_0}\ \Gamma_0,$$ 
	In fact, since $\mathtt{e} = \mathtt{true}$, it must be the case that $\Gamma=\Gamma_0$.
	By assumption, $\Gamma\vdash(s,\theta)$, and the store is unchanged by the transition, so~\eqref{pres-store1}
	is satisfied. Another precondition of \textsc{IfTypable1} is that 
	$$\Gamma_0\vdash\mathtt{m_1}\ |_{C_1}\ \Gamma_1.$$
	This gives us~\eqref{pres-type1} directly, and all that remains to show is
	that $\Gamma'\sqsubseteq \Gamma_1$ for~\eqref{pres-env1}.

	$\Gamma'=merge(\Gamma_0,\Gamma_1,\Gamma_2)$, so the elements of $\Gamma'$
	are the intersection of the corresponding elements in $\Gamma_1$ and
	$\Gamma_2$, and in particular 
	$$\forall \{id:T'\}\in\Gamma'\qdot \exists T_1 \qdot \{id:T_1\}\in\Gamma_1,$$
   	where $T' =	T_1\cap\Gamma_2(id)$. Lemma~\ref{lm:intersect} tells us that the
	intersection of two types is a supertype to both types. We can use this to
	show that $T'\succeq T_1$, and 
	derive the conclusion of Definition~\ref{eqn:strength}, and thus that
	$\Gamma'\sqsubseteq\Gamma_1$.~\eqref{pres-env1} is thus satisfied, and hence
	$\Phi(m,s,\theta,m',s',\theta')$.

  \end{case}

  \begin{case}[If3]
	Similar to \textsc{If2}.
  \end{case}

  \begin{case}[If4]
	$m'$ is immediately typable using rule \textsc{IfTypable1}, given that
	$e$, $m_1$ and $\epsilon$ are all typable. The store is unchanged so,
	as in the previous cases, we have that $\Gamma\vdash(s,\theta)$.
  \end{case}

  \begin{case}[While]
	Given that $e$, $m_1$ and $\mathtt{while\ (e)\ \{m_1\}}$ are all typable by the precondition
	of the typability judgement of $m$, we can immediately deduce that $m'$ is 
	typable by rules \textsc{SeqTypable} and \textsc{IfTypable2}.
  \end{case}

  \begin{case}[For]
	Similar to \textsc{While}.
  \end{case}

\end{proof}
