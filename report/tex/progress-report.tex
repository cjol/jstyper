% Draft #1

\documentclass{article}
\usepackage[T1]{fontenc}
\usepackage[super]{nth}
\usepackage[all]{nowidow}
\usepackage{amssymb}
\begin{document}

	\vfil 
	\centerline{\large Computer Science Project Progress Report}
	\vspace{0.2in}
	\centerline{\Large Building a Gradual Type System for JavaScript }
	\vspace{0.1in}
	\centerline{\large Christopher J.O. Little, Emmanuel College}
	\centerline{\large \texttt{cl554@cam.ac.uk}}

	\vspace{0.3in}

	\noindent{\bf Project Supervisor:} Dr Kathryn E. Gray
	\vspace{0.2in}

	\noindent{\bf Director of Studies:} Dr Jonathan Hayman
	\vspace{0.2in}\noindent 
	
	\noindent{\bf Project Overseers:} Dr~Markus~Kuhn \& Dr~Neal~Lathia

	\section*{Introduction}

	At this point in my project, I am confident that I am sufficiently on
	schedule, and that the project will be successfully completed on
	time. The table below summarises my completed goals in relation to the
	features of JavaScript outlined in my proposal. Most of the remaining
	features should be complete very soon (loops, for example, should present
	no particular complexity from a typing perspective), and so I anticipate
	having time to complete some of the extension tasks outlined in my
	proposal. The table columns have the following meaning:

	\begin{description}
		\item[\textit{spec}] --- Definition of Formal Type Specification \hfill \\

			The feature has been included in the formal specification of the
			type system constructed so far.
		
		\item[\textit{inf}] --- Implementation of Type Inference System \hfill \\

			The feature has been included in the implementation of the type
			inferencing system. For expressions, this means that the type of
			the expression is correctly inferred, and for other statements,
			this means that all expressions within that statement are
			correctly typed.
		
		\item[\textit{comp}] --- Implementation of Gradual Typing Compiler \hfill \\
			
			For expressions, this feature is correctly guarded after passing
			through the source-to-source compiler. In particular for
			expressions of primitive type, this means that a type-check is
			inserted before the expression is used, and for higher order
			expression types (object or function), this means that guard and
			mimic wrappers are correctly created around the expression. For
			statements, all expressions contained in the statement are
			correctly guarded.
		
		\item[\textit{proof}] \hfill \\
		
			The relevant cases in the proof of type soundness have been written
			up.

	\end{description}

	\section{Goals Achieved}

	\begin{table}[ht]
		\begin{tabular}{lcccc}
		                                  & \textit{spec} & \textit{inf} & \textit{comp} &  \textit{proof} \\ 
		\hline
		Primitive constants               & \checkmark{}  & \checkmark{} & \checkmark{}  &  \checkmark{}   \\
		seq (\texttt{\ldots; \ldots})     & \checkmark{}  & \checkmark{} & \checkmark{}  &  \checkmark{}   \\
		block (\texttt{\{\ldots\}})     & \checkmark{}  & \checkmark{} & \checkmark{}  &  \checkmark{}   \\
		\texttt{if \ldots~else}           & \checkmark{}  & \checkmark{} & \checkmark{}  &  \checkmark{}   \\
		Arithmetic Operators 
			(\texttt{+},\texttt{-},
			\texttt{*},\texttt{/})        & \checkmark{}  & \checkmark{} & \checkmark{}  &  \checkmark{}   \\
		Logical Operators 
			(\texttt{!}, \texttt{\&{}\&},
				 \texttt{||})             & \checkmark{}  & \checkmark{} & \checkmark{}  &  \checkmark{}   \\
		Shorthand Assignments 
			(\texttt{+=})                 & \checkmark{}  & \checkmark{} & \checkmark{}  &  \checkmark{}   \\
		Assignment (\texttt{=})           & \checkmark{}  & \checkmark{} & -             &  \checkmark{}   \\
		Variable references               & \checkmark{}  & \checkmark{} & -             &  \checkmark{}   \\
		Object Literals                   & \checkmark{}  & \checkmark{} & \checkmark{}  &                 \\
		Function Definition               & \checkmark{}  & \checkmark{} & \checkmark{}  &                 \\
		Function Calls 		              & \checkmark{}  & \checkmark{} &               &                 \\
		Object property access            & \checkmark{}  & \checkmark{} & 	             &                 \\
		\texttt{while}                    &               &              &               &                 \\
		\texttt{for}                      &               &              &               &                 \\
		\texttt{continue}                 &               &              &               &                 \\
		\texttt{break}                    &               &              &               &                 \\
		\end{tabular}
	\end{table}

	`-' indicates partial completion. In both cases, the feature works for
	primitive types, but not yet for higher order types.

	\section{Changes to Project}

		There are few significant changes to report. Firstly, I have decided that
		treating primitive types polymorphically is beyond the scope of this
		project. In particular, this means that my system will not be able to
		type expressions such as \texttt{``Hello '' + ``World''}, as this
		would require the operands of \texttt{+} to be typed either as
		integers or as strings. Additionally, this means that I will not be
		able to type-check function calls which have arguments omitted (since
		they are effectively being called with a parameter of type
		\texttt{undefined}).

	\section{Extension tasks}

		I have already largely implemented the ability to modify an object
		type by adding properties dynamically. I will not tackle the problem
		of removal of object properties, since this could only be done with
		control flow analysis. Array types, although useful, also likely
		present more complexity than I have time to handle. Their presence
		could still be handled (with gradual typing) as a kind of `black box',
		where every array retrieval is handled by introducing a new dynamic
		type. Prototype inheritance also presents a complex problem, but one
		which I believe I will be able to solve. This is such a common feature
		in production JavaScript code that it would be very beneficial to
		integrate. It may yet be that large commercial test suites will not be
		suitable for type-checking by my system, but I am confident I will be
		able to find fragments of production JavaScript which are. These will
		present interesting benchmarking opportunities.

	\section{Adjusted Timetable}

		\subsection*{Weeks 15 \& 16 (\nth{30} Jan -- \nth{12} Feb)}

		Extend gradual typing compiler to handle higher order types (guard and
		mimic wrappers). Extend type inference system to handle loops.

		\subsection*{Weeks 17 \& 18 (\nth{13} Feb -- \nth{26} Feb)}

		Investigate appropriate ways of modelling prototypal inheritance
		within my type inference system. Implement, or at least document
		promising avenues.

		\subsection*{Weeks 19 -- 23 (\nth{27} Feb -- \nth{26} Mar)}

		Finalise test suite, and obtain appropriate fragments of production
		JavaScript. Evaluate performance and additional memory overhead of
		compiled code.
		
		\subsection*{Weeks 24 -- 28 (\nth{27} Mar -- \nth{23} Apr)}

		Write dissertation and refine after feedback before submission.

\end{document}